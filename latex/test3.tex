% test3.tex
% More comprehensive example.
% Simon Heath
% 25/09/2002

\documentclass{article}
% begins preamble

\title{An Example Document}  % title
\author{Simon Heath} % author, of course.
\date{}  % I think this gives today's date; if not, just fill in the {}

\newcommand{\ip}[2]{(#1, #2)}
% Defines \ip{arg1}{arg2} to mean (arg1, arg2)

%\newcommand{\ip}[2]{\langle #1 | #2\rangle}
% Commented-out, alternative definition of \ip

\begin{document}
% Ends preamble and begins text.

\maketitle  % Produces the title.

This is an example input file.  Comparing it with the output it generates
can show you how to produce a simple document of your own.

\section{Ordinary Text}
% Produces a section heading.  Subsections are begun with similar
% \subsection and \subsubsection commands

The end of words and sentences are marked    by    spaces.  It doesn't
matter how many spaces   you    type, 1 is as good as 100.  The end of
a line counts of a space.

One or more blank lines denote the end of a paragraph.  Since any number
of consecutive spaces are treated like a single one, the formatting of
the input file makes no differente to \LaTeX\  % Generates LaTeX logo
but it makes a difference to you.  When you use \LaTeX, making your
input file easy to read as possible will be a great help as you write
and change your document.  The sample file shows how you can add
comments to your own input file.

Because printing is different from typewriting, there are a number of things
you have to do differently when prepareing an input file than if
you were just typing a document directly.  Quotation marks like ``this''
have to be specially, as do quotes within quotes:
``\,`this'  % \, seperates the single and double quotes.
is what I just wrote, not `that'\,''.

Dashes come in three size:  an intra-word dash, a medium-range dash for
number ranges like 1--2, and a punctuation dash ---like this.

A sentence-ending space should be larger than the space between words in
a sentence.  You sometimes have to type special commands in conjunction
with punctuation characters to get this right, as in the following sentence.  
Gnats, gnus, etc.\     % the \ makes an inter-word space
all begin with G\@.    % \@ marks end-of-sentence punctuation.  
You should check your spaces after periods when reading your output to 
make sure you haven't forgotten any special cases.  Generating an 
ellipsis \ldots\   
% the \ is needed after \ldots because TeX ignores
% spaces after command names made from \+letters.  So it makes sure it
% adds the spaces.
with the right spacing around the periods requires a special command.

Text is usually emphasised like \emph{this}.
\begin{em}A long section of text can also be emphasized like this.  Text 
within a segment can be given \emph{additional} emphasis.
\end{em}

It is sometimes necessary to prevent \LaTeX\ from breaking a line where it
might otherwise do so.  This may be at a space, as between the ``Mr.'' and 
``Jones'' in ``Mr.~Jones'', % ~ produces an unbreakable interword space
or within a word---especially when the word is a symbol like
\mbox{\emph{itemnum}}
that makes little sense when hyphenated across lines.

Footnotes\footnote{This is an example of a footnote} pose no problem.

\LaTeX\ is good at typesetting mathematical formulae like
\( x-3y + z = 7 \) or 
\( a+{1} > x^{2n} + y{2n} > x' \) or
\( \ip{A}{B} = \sum_{i} a_{i} b_{i} \).  % We use our above-defined function
The spaces you type in a formula are ignored.  Remember that a letter like
$x$  % $...$ is the same as \( ... \)
is a formula when it denotes a mathematical symbol, and it should be typed
as one.

\section{Displayed Text}

Text is displayed by intenting it from the left margin.  Quotations are 
commonly displayed.  There are short quotations:
\begin{quote}
This is a short quotation.  It consists of a single paragraph of text.
See how it is formatted.
\end{quote}
and longer ones---
\begin{quotation}
This is a longer quotation.  It consists of two paragraphs of text, neither
of which are very interesting.

This is the second paragraph.  Move along now.  Nothing to see.
\end{quotation}

Another frequently-displayed structure is a list.  The following is an example
of an \emph{itemized} list.

\begin{itemize}
\item This is the first item of an itemized list.  Each item is marked with
a ``tick''.
\item You don't have to worry about what kind of tick mark is used.
\item This is the third item of the list.  It contains another list nested
inside it.  The inner list is an \emph{enumerated} list.
\begin{enumerate}
\item This is the first item of an enumarated list nested inside the itemized
list.
\item This is the second item of the inner list.  \LaTeX\ allows you to nest
lists deeper than you really should.
\end{enumerate}
This is the rest of the second item of the outer list.
\item This is the fourth item.
\end{itemize}

You can even display poetry.
\begin{verse}
There is an environment for verse \\  % the \\ lets you seperate lines
whose features some poets             % within a stanza.
will curse.

% one or more blank lines seperate stanzas.
For instead of making\\
Them do \emph{all} line breaking,\\
It allows them to put too many words on a line when they'd rather be forced to
be terse.
\end{verse}

Mathematical formulas may also be displayed.  A displayed formula is one line
long; multiline formulas require special formatting instructions.
\[ \ip{\Gamma}{\psi'} = x'' + y^{2} + z_{i}^{n} \]

Don't start a paragraph with a displayed equation, nor make one a paragraph
by itself.

\end{document}
