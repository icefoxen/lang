\documentclass[10pt,letterpaper,notitlepage,oneside]{article}
% The [] options are the LaTeX defaults for 'article' classes.
% other docclasses are report and book.  They have different defaults.

\usepackage{verbatim}  % lets you use \begin{comment} and \end{comment}
% Other significant packages include doc, excale, fontenc, ifthen, latexsym, 
% makeidx, syntonly, inputenc

\pagestyle{plain}
% plain prints page numbers in the middle-bottom of the page.
% headings prints the current chapter heading and page number on top, while
% the footer is empty.
% empty sets both header and footer to empty.

\usepackage{syntonly}
%\syntaxonly
% Uncomment the above line to tell it not to produce actual output, just check
% the file.

\hyphenation{FORTRAN hy-phen-a-tion}
% Forces hyphenation behavior.  Makes it so it only hyphenates the listed
% words at the selected points.  In this instance, it will not hyphenate
% FORTRAN, Fortran or fortran at all.


\author{Simon N. Heath}
\title{Example 4ish}

\begin{document}

\maketitle 

\emph{This file is meant to be read while looking at the source!}
\emph{Lotsa handy comments there.}

% You can put {} after a command to keep it from eating up whitespace
% You can also put a \ after it, id est \TeX\
I read that Knuth divides the people working with \TeX{} into
\TeX{}nicians and \TeX perts

You can \textsl{lean} on me!  Is this the same as \emph{emphasis}?  Nope!

Please start a newline here--\newline
Thank you!  One more please?\\
Thanx! Today is \today{}, by the way.

This is another
\begin{comment}
Other than %, you can insert long comments like this.
You have to add \usepackage{verbatim} to the doc preamble,
though.
\end{comment}
example of adding comments to the source file.

This is an included file!  Yay!

What should we \emph{put} in this included file though?

I know!  How 'bout the date!  Yay!  \today

% Includes another file.  Starts a new page before including.

FOOBAR!

This is another included file!  Yay!

Woohoo!!  W00t!  Breet!  Squack!  *frink!*

% This does the exact same, but doesn't start a new page.
% ...or it does, but it shouldn't.  Ah well.
% Doesn't need the .tex at the end!

You can force new lines!\\

You can force new pages!\newpage

You can force linebreaks!\linebreak[4]  % Arg can be a number 0-4... how hard
% you want to force the linebreak.  LaTeX will try to ignore it if it'll
% look reely bad, 4 forces it to obey.

You can force nolinebreaks\nolinebreak[4]
% Can also do \nopagebreak[]   ...Args are optional on all of these.
% The '...break' commands make LaTeX still try to justify the text.

% You can give the commands \sloppy and \fussy to make LaTeX less
% stringent about justifying words.  Default is \fussy.  \sloppy just turns
% it off.

Several words can be kept together on one line like this: \mbox{What a 
wonderful weird world this is...} \\
There should be a \fbox{box} drawn \fbox{here}.

Hey, there are FOUR types of dashes...
daughter-in-law, Q-tip\\
pages 14--41\\
yes---or no?\\
$0$, $1$, and $-1$\\
% I think the $...$ show it's an equation.

There's two kinds of tildes:\\
http://www.rich.edu/\~{}bush \\  % little tilde
http://www.rich.edu/$\sim$demo \\  % Big tilde

You can add a degree symbol, even.  It is now $20,^{\circ}\mathrm{C}$ and
I like air-conditioning.

Ellipsis\ldots

Time for accents!\\
H\^o{}tel, na\"\i ve, \'el\`eve, sm\o rrebr\o d, !`Se\~norita!,
Sch\"onbrunner Schlo\ss{} Stra\ss e





\end{document}
